\documentclass[10p]{report}
\usepackage[utf8]{inputenc}
\usepackage[greek]{babel}
\usepackage{alphabeta}
\usepackage[LGR, T1]{fontenc}
\usepackage{textpos}
\usepackage{dejavu}
\usepackage[left=1.2in, right=1.2in]{geometry}
\usepackage{amsmath}
\usepackage{listings}
\usepackage[hidelinks]{hyperref}

\begin{document}

% Title
\begin{textblock}{7.5}(1.99,-1.5)	
\begin{center}
\textbf{\Large{Παράλληλα και Διανεμημένα Συστήματα}}\\
\textbf{\large{Αναφορά \latintext{Εργασία 4}}} \\
\normalsize{Παρασκευόπουλος Ιάσων (ΑΕΜ 8410)}
\end{center}
\end{textblock}

\section*{Γενικά για τον αλγόριθμο}

Για την υλοποίηση του προβλήματος \latintext{pagerank} με μέθοδο επίλυσης \latintext{Gauss-Seidel}
ακολουθήθηκε η λογική του κώδικα που δόθηκε ως παράδειγμα στο πακέτο ανοιχτού
κώδικα της εκφώνησης. 

Η μέθοδος Gauss-Seidel είναι μία επαναλητπική τεχνική που λύνει ένα τετραγωνικό
σύστημα γραμμικών εξισώσεων της μορφής $ Ax=b $. Έχοντας μία αρχική εκτίμηση των τιμών
του $ x $ σε κάθε επανάληψη υπολογίζεται η επόμενη τιμή του χρησιμοποιώντας τον
ακόλουθο τύπο $ x_i^{k+1} = \dfrac{1}{a_{aii}}(b_i - \sum_{j=1}^{i-1} a_{ij}x_j^{k+1} - \sum_{j=i+1}^{n} a_{ij}x_j^k)$.
Για να μπορέσει να ακολουθηθεί αυτή η επίλυση
εφαρμόζεται η αλγεβρική μορφή του pagerank $ R = (I - dM)^{-1}\dfrac{1-d}{N} $. 

\section*{Μέθοδος παραλληλοποίησης}

Παρατηρώντας τον τύπο του Gauss-Seidel είναι φανερό ότι δεν είναι ιδανικός για
παραλληλοποίηση μιας και σε κάθε επανάληψη $ k $ για κάθε στοιχείο $ x_i $ για όλα τα
στοιχεία με $ j < i $ χρησιμοποιούνται οι τιμές $ x_j^{k}$ και όχι $ x_j^{k-1}$.
Δηλαδή χρησιμοποιούνται και τιμές της τρέχουσας επανάληψης και όχι μόνο της προηγούμενης, με αποτέλεσμα
να είναι αρκετά σειριακός ο αλγόριθμος. Όμως γίνεται να αντιμετωπιστεί αυτό το
πρόβλημα αν τα στοιχεία ομαδοποιηθούν με τέτοιο τρόπο ώστε σε κάθε ομάδα κανένα
στοιχείο να μην εξαρτάται από την τιμή ενός άλλου στοιχείου της ομάδας.
Δηλαδή για παράδειγμα αν ήταν 20 τα στοιχεία μας να γίνει αναδιάταξη τους έτσι
ώστε τα πρώτα δέκα στις εξισώσεις τους να περιέχουν μόνο στοιχεία από τα υπόλοιπα
δέκα. Έτσι κανένα στοιχείο δεν θα χρειάζεται την νέα τιμή κάποιου άλλου και θα
μπορούν να υπολογιστούν ταυτόχρονα οι νέες τους τιμές χρησιμοποιώντας τις τιμές
των υπόλοιπων στοιχείων από την προηγούμενη επανάληψη.

Αυτό το πρόβλημα είναι γνωστό και αναγάγεται στον χρωματισμό των γράφων. Στην
περίπτωση του pagerank κάθε σελίδα είναι ένας κόμβος και αν έχει ή δέχεται link
από κάποια άλλη τότε υπάρχει ακμή μεταξύ τους. Σκοπός του χρωματισμού του γράφου
είναι να χρωματιστεί κάθε κόμβος με ένα χρώμα τέτοιο ώστε να μην είναι το ίδιο
με κανένα από τα χρώματα των γειτόνων του. 

Έχουν αναπτυχθεί αρκετοί αλγόριθμοι που λύνουν το συγκεκριμένο πρόβλημα και για
την εργασία υλοποιήθηκε ένας άπληστος αλγόριθμος, ο οποίος λειτουργεί
με τον εξής τρόπο. Αρχικά δίνεται το χρώμα 0 στον πρώτο κόμβο και στην συνέχεια
για κάθε κόμβο σημειώνονται ως μη διαθέσιμα τα χρώματα των γειτόνων του και του
αναθέτεται το χρώμα με την μικρότερη τιμή από τα υπόλοιπα που είναι διαθέσιμα. 
\newpage
\begin{lstlisting}
color[0] = 0
for vertex in vertexes
    for neighbor in neihbors
        available[color[neighbor]] = 0
    for col in available
        if available[col]
            color[vertex] = col
\end{lstlisting}

Ο αλγόριθμος αναφέρει ότι ο μέγιστος αριθμός χρωμάτων που αναθέτονται είναι
$ d + 1 $ όπου $ d $ είναι ο βαθμός του γράφου. Αλλά αντί να αρχικοποιηθούν τα διαθέσιμα
χρώματα με αυτή την τιμή, για να μην καθυστερήσει η διαδικασία εφαρμόστηκε μία
δυναμική προσέγγιση όπου τα διαθέσιμα χρώματα εκκινούν από 100 και αν δεν
υπάρχει άλλο διαθέσιμο τότε προσθέτονται άλλα 100 σε αυτά.
Στην συνέχεια αφού έχουν βρεθεί τα χρώματα των κόμβων ομαδοποιούνται ώστε να
επεξεργαστούν σε ομάδες. Για την ομαδοποίηση τους απλά ταξινομήθηκαν τα χρώματα
των κόμβων σε αύξουσα σειρά.

Για την ταξινόμηση χρησιμοποιήθηκε μία παραλλαγή της quicksort όπου μαζί με τα
χρώματα άλλαζαν θέση και τα indexes των κόμβων. Επίσης για partitioning στην
quicksort δεν χρησιμοποιήθηκε ο αλγόριθμος του 
\href{https://en.wikipedia.org/wiki/Quicksort#Hoare_partition_scheme}{\underline{Hoare}}
αλλά η \underline{fat partition}
γιατί ειδικά στα αρχικά χρώματα πολλά στοιχεία θα έχουν την ίδια τιμή. Μετά την
σωστή κατανομή των στοιχείων αναδιατάσσεται ο πίνακας γειτνίασης. 

Αφού αναδιαταχτεί ο πίνακας γειτνίασης για κάθε ομάδα χρησιμοποιείται ο
αλγόριθμος του Gauss-Seidel ταυτόχρονα. 

\section*{Μετρήσεις και σχόλια}

\section*{Έλεγχος ορθότητας}

Για τον έλεγχο ορθότητας της συγκεκριμένης υλοποίησης γίνεται σύγκριση των αποτελεσμάτων 
του αλγορίθμου με τα αποτελέσματα της υλοποίησης ανοιχτού κώδικα που δόθηκε στην εκφώνηση. Για 
κάθε dataset που χρησιμοποιήθηκε πρώτα έγινε ο υπολογισμός του pagerank στην 
matlab και εξάχθηκαν τα αποτελέσματα σε ένα αρχείο, από το οποίο διαβάζεται το 
διάνυσμα του pagerank και συγκρίνεται στοιχείο με στοιχείο με την υλοποίηση 
που αναπτύχθηκε στα πλαίσια της εργασίας. 

\end{document}

