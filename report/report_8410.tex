\documentclass[10p]{report}
\usepackage[utf8]{inputenc}
\usepackage[greek]{babel}
\usepackage{alphabeta}
\usepackage[LGR, T1]{fontenc}
\usepackage{textpos}
\usepackage{dejavu}
\usepackage[left=1.2in, right=1.2in]{geometry}
\usepackage{amsmath}
\usepackage{listings}
\usepackage[hidelinks]{hyperref}
\usepackage{changepage}

\begin{document}

% Title
\begin{textblock}{7.5}(1.99,-1.5)	
\begin{center}
\textbf{\Large{Παράλληλα και Διανεμημένα Συστήματα}}\\
\textbf{\large{Αναφορά \latintext{Εργασία 4}}} \\
\normalsize{Παρασκευόπουλος Ιάσων (ΑΕΜ 8410)}
\end{center}
\end{textblock}

\section*{Γενικά για τον αλγόριθμο}

Για την υλοποίηση του προβλήματος \latintext{pagerank} με μέθοδο επίλυσης \latintext{Gauss-Seidel}
ακολουθήθηκε η λογική του κώδικα που δόθηκε ως παράδειγμα στο πακέτο ανοιχτού
κώδικα της εκφώνησης. 

Η μέθοδος Gauss-Seidel είναι μία επαναλητπική τεχνική που λύνει ένα τετραγωνικό
σύστημα γραμμικών εξισώσεων της μορφής $ Ax=b $. Έχοντας μία αρχική εκτίμηση των τιμών
του $ x $ σε κάθε επανάληψη υπολογίζεται η επόμενη τιμή του χρησιμοποιώντας τον
ακόλουθο τύπο $ x_i^{k+1} = \dfrac{1}{a_{aii}}(b_i - \sum_{j=1}^{i-1} a_{ij}x_j^{k+1} - \sum_{j=i+1}^{n} a_{ij}x_j^k)$.
Για να μπορέσει να ακολουθηθεί αυτή η επίλυση
εφαρμόζεται η αλγεβρική μορφή του pagerank $ R = (I - dM)^{-1}\dfrac{1-d}{N} $. 

\section*{Μέθοδος παραλληλοποίησης}

Παρατηρώντας τον τύπο του Gauss-Seidel είναι φανερό ότι δεν είναι ιδανικός για
παραλληλοποίηση μιας και σε κάθε επανάληψη $ k $ για κάθε στοιχείο $ x_i $ για όλα τα
στοιχεία με $ j < i $ χρησιμοποιούνται οι τιμές $ x_j^{k}$ και όχι $ x_j^{k-1}$.
Δηλαδή χρησιμοποιούνται και τιμές της τρέχουσας επανάληψης και όχι μόνο της προηγούμενης, με αποτέλεσμα
να είναι αρκετά σειριακός ο αλγόριθμος. Όμως γίνεται να αντιμετωπιστεί αυτό το
πρόβλημα αν τα στοιχεία ομαδοποιηθούν με τέτοιο τρόπο ώστε σε κάθε ομάδα κανένα
στοιχείο να μην εξαρτάται από την τιμή ενός άλλου στοιχείου της ίδιας ομάδας.
Δηλαδή για παράδειγμα αν ήταν 20 τα στοιχεία θα έπρεπε να γίνει αναδιάταξη τους έτσι
ώστε τα πρώτα δέκα στις εξισώσεις τους να περιέχουν μόνο στοιχεία από τα υπόλοιπα
δέκα. Έτσι κανένα στοιχείο δεν θα χρειάζεται την νέα τιμή κάποιου άλλου και θα
μπορούν να υπολογιστούν ταυτόχρονα οι νέες τους τιμές χρησιμοποιώντας τις τιμές
των υπόλοιπων στοιχείων από την προηγούμενη επανάληψη.

Αυτό το πρόβλημα είναι γνωστό και αναγάγεται στον χρωματισμό των γράφων. Στην
περίπτωση του pagerank κάθε σελίδα είναι ένας κόμβος και αν έχει ή δέχεται link
από κάποια άλλη τότε υπάρχει ακμή μεταξύ τους. Σκοπός του χρωματισμού του γράφου
είναι να χρωματιστεί κάθε κόμβος με ένα χρώμα τέτοιο ώστε να μην είναι το ίδιο
με κανένα από τα χρώματα των γειτόνων του. 

Έχουν αναπτυχθεί αρκετοί αλγόριθμοι που λύνουν το συγκεκριμένο πρόβλημα και για
την εργασία υλοποιήθηκε ένας άπληστος αλγόριθμος, ο οποίος λειτουργεί
με τον εξής τρόπο. Αρχικά δίνεται το χρώμα 0 στον πρώτο κόμβο και στην συνέχεια
για κάθε κόμβο σημειώνονται ως μη διαθέσιμα τα χρώματα των γειτόνων του και του
αναθέτεται το χρώμα με την μικρότερη τιμή από τα υπόλοιπα που είναι διαθέσιμα. 
\newpage
\begin{lstlisting}
color[0] = 0
for vertex in vertexes
    for neighbor in neihbors
        available[color[neighbor]] = 0
    for col in available
        if available[col]
            color[vertex] = col
\end{lstlisting}

Ο αλγόριθμος αναφέρει ότι ο μέγιστος αριθμός χρωμάτων που αναθέτονται είναι
$ d + 1 $ όπου $ d $ είναι ο βαθμός του γράφου. Αλλά αντί να αρχικοποιηθούν τα διαθέσιμα
χρώματα με αυτή την τιμή, για να μην καθυστερήσει η διαδικασία εφαρμόστηκε μία
δυναμική προσέγγιση όπου τα διαθέσιμα χρώματα εκκινούν από 100 και αν δεν
υπάρχει άλλο διαθέσιμο τότε προσθέτονται άλλα 100 σε αυτά.
Στην συνέχεια αφού έχουν βρεθεί τα χρώματα των κόμβων ομαδοποιούνται ώστε να
επεξεργαστούν σε ομάδες. Για την ομαδοποίηση τους απλά ταξινομήθηκαν τα χρώματα
των κόμβων σε αύξουσα σειρά.

Για την ταξινόμηση χρησιμοποιήθηκε μία παραλλαγή της quicksort όπου μαζί με τα
χρώματα άλλαζαν θέση και τα indexes των κόμβων. Επίσης για partitioning στην
quicksort δεν χρησιμοποιήθηκε ο αλγόριθμος του 
\href{https://en.wikipedia.org/wiki/Quicksort#Hoare_partition_scheme}{\textit{Hoare}}
αλλά η \textit{fat partition}
γιατί ειδικά στα αρχικά χρώματα πολλά στοιχεία θα έχουν την ίδια τιμή. Μετά την
σωστή κατανομή των στοιχείων αναδιατάσσεται ο πίνακας γειτνίασης. 

Αφού αναδιαταχτεί ο πίνακας γειτνίασης για κάθε ομάδα γίνεται παράλληλα η ανανέωση 
των στοιχείων της. Πιο συγκεκριμένα χρησιμοποιήθηκε μία omp parallel for στην εξωτερική 
επανάληψη του σειριακού αλγορίθμου.

Παρόλα αυτά παρατηρήθηκε ότι αν δεν εφαρμοστεί η μέθοδος του graph coloring και 
γίνει κατευθείαν στον μη αναδιατεταγμένο πίνακα γειτνίασης η παράλληλη υλοποιήση
τότε το τελικό αποτέλεσμα είναι σχεδόν όμοιο με τον σειριακό αλγόριθμο. Οπότε 
στην υλοποίηση προστέθηκε και η περίπτωση να γίνει ο παραλληλισμός στον αρχικό πίνακα 
γειτνίασης με μία ομάδα που περιέχει όλα τα στοιχεία του.


\section*{Μετρήσεις και σχόλια}

Για την αξιολόγηση των αλγορίθμων έγιναν μετρήσεις στα παρακάτω dataset χρησιμοποιώντας 
τον σειριακό αλγόριθμο στην matlab, τον σειριακό αλγόριθμο στην c που αναπτύχθηκε 
για την υλοποιήση του παράλληλου, τον παράλληλο χωρίς την χρήση graph coloring για 
2,4 και 8 threads, τον σειριακό με την χρήση graph coloring και τον παράλληλο με 
την χρήση graph coloring για 2,4 και 8 threads. 

\begin{table}[htbp]

\footnotesize{
        \begin{center}
        \begin{tabular}{| c | c | c |} \hline
                Dataset & Nodes & Edges \\ \hline
                cs-stanford & 9914 & 36854 \\ %\hline 
                web-Google & 875,713 & 5,105,039 \\ %\hline
                web-BerkStan & 685,230 & 7,600,595 \\ %\hline
                wiki-Talk & 2,394,385 & 5,021,410 \\ %\hline
                cit-Patents & 3,774,768 & 16,518,948 \\ %\hline
                soc-pokec-relationships & 1,632,803 & 30,622,564 \\ \hline
        \end{tabular}
                \caption{datasets}
        \end{center}
}
\end{table}

%Παραπάνω πληροφορίες για τα dataset μπορούν να βρεθούν 
%\href{https://snap.stanford.edu/data/}{\textit{εδω}}.

%Στον παρακάτω πίνακα παραθέτονται τα αποτελέσματα από τις μετρήσεις.
\begin{table}[htbp]
\normalsize{
        %\begin{center}
        \begin{adjustwidth}{-0.5in}{}
        \begin{tabular}{| c | c | c | c | c | c | c |} \hline
                method & cs-stanford & web-Google & web-BerkStan & wiki-Talk & cit-Patents & soc-pokec\\ \hline
                matlab &
                \begin{tabular}{c} 0.007658\\ 0.000193\\ 37\\ 0\\
                \end{tabular} &
                \begin{tabular}{c} 2.211056\\ 0.0598\\ 36\\ 0\\
                \end{tabular} &
                \begin{tabular}{c} 0.946292\\  0.0225\\ 41\\ 0\\
                \end{tabular} &
                \begin{tabular}{c} 0.714777\\ 0.0446\\ 16\\ 0\\
                \end{tabular} &
                \begin{tabular}{c} 4.003598\\ 0.3080\\ 13\\ 0\\
                \end{tabular} &
                \begin{tabular}{c} 11.610687\\ 0.2903\\ 40\\ 0\\
                \end{tabular} \\
                \hline
                serial &
                \begin{tabular}{c} 0.005100\\ 0.000137\\ 37\\ 4.15e-8\\
                \end{tabular} &
                \begin{tabular}{c} 2.085312\\ 0.057925\\ 36\\ 3.92e-8\\
                \end{tabular} &
                \begin{tabular}{c} 0.750975\\ 0.018316\\ 41\\ 5.08e-8\\
                \end{tabular} &
                \begin{tabular}{c}0.621157 \\ 0.038822\\ 16\\ 5e-9\\ 
                \end{tabular} &
                \begin{tabular}{c} 3.405085\\ 0.261929\\ 13\\ 9.25e-8\\ 
                \end{tabular} &
                \begin{tabular}{c} 11.457829\\ 0.286445\\ 40\\ 2.1e-8\\ 
                \end{tabular} \\
                \hline
                par 2 t&
                \begin{tabular}{c} 0.003707\\ 0.000100\\ 37\\ 4.43e-8\\
                \end{tabular} &
                \begin{tabular}{c} 1.515956\\ 0.042109\\ 36\\ 2.287e-7\\
                \end{tabular} &
                \begin{tabular}{c} 0.568299\\ 0.013860\\ 41\\ 6.43e-8\\
                \end{tabular} &
                \begin{tabular}{c} 0.478145\\ 0.029884\\ 16\\ 1.488e-7\\ 
                \end{tabular} &
                \begin{tabular}{c} 2.850138\\ 0.219241\\ 13\\ 9.25e-8\\ 
                \end{tabular} &
                \begin{tabular}{c} 8.879531\\ 0.253700\\ 35\\ 4.02e-8\\ 
                \end{tabular} \\
                \hline
                par 4 t&
                \begin{tabular}{c} 0.004328\\ 0.000116\\ 37\\ 5.04e-8\\
                \end{tabular} &
                \begin{tabular}{c} 1.420369\\ 0.039454\\ 36\\ 3.007e-8\\
                \end{tabular} &
                \begin{tabular}{c} 0.595234\\ 0.014517\\ 41\\ 8.47e-8\\
                \end{tabular} &
                \begin{tabular}{c} 0.527285\\ 0.032955\\ 16\\ 3.038e-7\\ 
                \end{tabular} &
                \begin{tabular}{c} 2.253026\\ 0.187752\\ 12\\ 9.25e-8\\ 
                \end{tabular} &
                \begin{tabular}{c} 5.787917\\ 0.199583\\ 29\\ 2.433e-7\\ 
                \end{tabular}  \\
                \hline
                par 8 t&
                \begin{tabular}{c} 0.005434\\ 0.000146\\ 37\\ 5.32e-8\\
                \end{tabular} &
                \begin{tabular}{c} 1.477376\\ 0.038878\\ 38\\ 4.013e-8\\
                \end{tabular} &
                \begin{tabular}{c} 0.530801\\ 0.012946\\ 41\\ 1.558e-7\\
                \end{tabular} &
                \begin{tabular}{c} 0.440316\\ 0.027519\\ 16\\ 7.31e-7\\ 
                \end{tabular} &
                \begin{tabular}{c} 1.964855\\ 0.178622\\ 11\\ 9.25e-8\\ 
                \end{tabular} &
                \begin{tabular}{c} 6.004861\\ 0.187651\\ 32\\ 3.003e-7\\ 
                \end{tabular}  \\
                \hline
                serial g-c&
                \begin{tabular}{c}0.005319 \\ 0.000147\\ 36\\ 1.375e-7\\
                \end{tabular} &
                \begin{tabular}{c}2.697137 \\ 0.069157\\ 39\\ 4.23e-7\\
                \end{tabular} &
                \begin{tabular}{c} 1.788035\\ 0.040637\\ 44\\ 3.607e-7\\
                \end{tabular} &
                \begin{tabular}{c} 0.849901\\ 0.053118\\ 16\\ 2.9257e-6\\ 
                \end{tabular} &
                \begin{tabular}{c} 2.827616\\ 0.314179\\ 9\\ 9.56e-8\\ 
                \end{tabular} &
                \begin{tabular}{c} 13.505419\\ 0.365011\\ 37\\ 1.959e-7\\ 
                \end{tabular}\\ 
                \hline
                par g-c 2 t &
                \begin{tabular}{c} 0.004880\\ 0.000135\\ 36\\ 1.375e-7\\
                \end{tabular} &
                \begin{tabular}{c} 1.842837\\ 0.047252\\ 39\\ 4.23e-7\\
                \end{tabular} &
                \begin{tabular}{c} 1.367416\\ 0.031077\\ 44\\ 3.607e-7\\
                \end{tabular} &
                \begin{tabular}{c} 0.579336\\ 0.036208\\ 16\\ 2.9257e-6\\ 
                \end{tabular} &
                \begin{tabular}{c} 2.230566\\ 0.247840\\ 9\\ 9.56e-8\\ 
                \end{tabular} &
                \begin{tabular}{c} 9.161404\\ 0.247605\\ 37\\ 1.959e-7\\ 
                \end{tabular}  \\
                \hline
                par g-c 4 t &
                \begin{tabular}{c} 0.005540\\ 0.000153\\ 36\\ 1.375e-7\\
                \end{tabular} &
                \begin{tabular}{c} 1.786479\\ 0.045807\\ 39\\ 4.23e-7\\
                \end{tabular} &
                \begin{tabular}{c} 1.692171\\ 0.038458\\ 44\\ 3.607e-7\\
                \end{tabular} &
                \begin{tabular}{c} 0.528391\\ 0.033024\\ 16\\ 2.9257e-6\\ 
                \end{tabular} &
                \begin{tabular}{c} 1.895726\\ 0.210636\\ 9\\ 9.56e-8\\ 
                \end{tabular} &
                \begin{tabular}{c} 8.645610\\ 0.233665\\ 37\\ 1.959e-7\\ 
                \end{tabular}  \\
                \hline
                par g-c 8 t &
                \begin{tabular}{c} 0.027576\\ 0.000766\\ 36\\ 1.375e-7\\
                \end{tabular} &
                \begin{tabular}{c} 1.844832\\ 0.047303\\ 39\\ 4.23e-7\\
                \end{tabular} &
                \begin{tabular}{c} 1.831261\\ 0.041619\\ 44\\ 3.607e-7\\
                \end{tabular} &
                \begin{tabular}{c} 0.527543\\ 0.032971\\ 16\\ 2.9257e-6\\ 
                \end{tabular} &
                \begin{tabular}{c} 1.852623\\ 0.205847\\ 9\\ 9.56e-8\\ 
                \end{tabular} &
                \begin{tabular}{c} 8.480382\\ 0.229199\\ 37\\ 1.959e-7\\ 
                \end{tabular}  \\
                \hline
        \end{tabular}
        \caption{μετρήσεις}
\end{adjustwidth}
}
%\end{center}
\end{table}
\newpage

Στον παραπάνω πίνακα για κάθε μέτρηση που έγινε αναγράφεται ο συνολικός χρόνος
για την ολοκλήρωση του αλγορίθμου, ο χρόνος ολοκλήρωσης μίας επανάληψης, ο αριθμός 
των επαναλήψεων και η ευκλείδεια απόσταση του αποτελέσματος με το αποτέλεσμα 
της matlab. 

Αρχικά από τα αποτελέσματα φαίνεται ότι ο σειριακός αλγόριθμος με την χρήση 
graph coloring έχει διαφορετικές επιδόσεις και αριθμό επαναλήψεων σε σύγκριση με
τον απλό σειριακό. Αυτό συμβαίνει λόγω την φύσης της μεθόδου του Gauss-Seidel όπου χρησιμοποιείται 
η νέα τιμή των στοιχείων σε κάθε επανάληψη οπότε αν αλλάξει η σειρά τους αλλάζει 
και η τιμή τους. Επίσης έχει και διαφορετικό error συγκριτικά με την matlab το οποίο 
όμως είναι της τάξεως του $ 10^7 $ οπότε μπορεί να θεωρηθεί αμελητέο. 

Σχετικά με τις παράλληλες υλοποιήσεις φαίνεται ότι υπάρχει επιτάχυνση συγκριτικά 
με τις σειριακές υλοποιήσεις με την χρήση ή όχι graph coloring. Σε αυτό το σημείο 
αξίζει να αναλυθούν πρώτα ξεχωριστά αυτές οι δύο υποπεριπτώσεις και στην συνέχεια 
συγκριτικά μεταξύ τους. 

\subsection*{Χωρίς graph-coloring}

Όσων αφορά την υλοποίηση χωρίς την χρήση graph coloring σε κάποιες περιπτώσεις 
παρατηρείται επιτάχυνση έως και 50\%. Όμως με το τίμημα η τιμή της ευκλείδειας απόστασης 
να μεγαλώνει όσο αυξάνονται τα threads και ο αριθμός των επαναλήψεων κάποιες φορές 
να μειώνεται. Επίσης η τιμή της ευκλείδειας απόστασης δεν ήταν ποτέ ίδια μεταξύ δύο 
συνεχόμενων μετρήσεων με τις ίδιες παραμέτρους εισόδου. 

Όλα τα παραπάνω είναι απόρροια της τυχαιότητας του παράλληλου gauss-seidel χωρίς 
graph coloring. Δηλαδή σε κάθε επανάληψη στα στοιχεία που χρειάζονται τις ανανεωμένες 
τιμές άλλων στοιχείων είναι πιθανό να μην έχει γίνει η ανανέωση τους γιατί μπορεί 
το thread που είναι υπεύθυνο για αυτή να μην έχει φτάσει σε αυτό το σημείο, οπότε 
χρησιμοποιούν τις τιμές της προηγούμενης επανάληψης. 

\subsection*{Με graph-coloring}

Εδώ παρατηρείται πάλι επιτάχυνση εως της τάξεως του 50\% σε σύγκριση με τον σειριακό
με την χρήση graph-coloring. Σε σύγκριση με τον σειριακό αλγόριθμο χωρίς graph-coloring 
στα περισσότερα dataset υπάρχει επιτάχυνση αλλά όχι το ίδιου βαθμού. 

Από την άλλη ο αριθμός των επαναλήψεων είναι σταθερός με τον σειριακό όπως και 
η τιμή της ευκλείδειας απόστασης για κάθε μέτρηση που έγινε. 

\subsection*{Συμπεράσματα}
Από τις μετρήσεις που έγιναν παρατηρείται ότι και στις δύο παράλληλες υλοποιήσεις 
σε κάποιες περιπτώσεις μετά από ένα σημείο όσο μεγάλωνε ο αριθμός των threads 
άρχιζε να αυξάνεται και ο χρόνος ανα επανάληψη. Επίσης και στις δύο υλοποιήσεις 
υπήρχε επιτάχυνση σε σύγκριση με την υλοποιήση της matlab. 

Τελικά τα συμπεράσματα που βγαίνουν από τις μετρήσεις είναι ότι σε σύνολο η 
υλοποίηση με graph coloring είναι πιο αργή από αυτήν χωρίς και στον σειρακό και 
στον παράλληλο αλγόριθμο. Αλλά η πρώτη είναι πιο σταθερή στην παράλληλη έκδοσή της 
σε σύγκριση με την πρώτη. 

Τέλος η έκδοση με graph coloring μπορεί να βελτιστοποιηθεί επιπλέον χρησιμοποιώντας 
έναν πιο βέλτιστο αλγόριθμο για τον χρωματισμό του γράφου και τροποποιώντας την 
υπάρχουσα υλοποίηση έτσι ώστε οι ομάδες χρωμάτων που έχουν λίγα στοιχεία να ενώνονται 
και να εκτελείται σειριακά η ανανέωση τους. Η δεύτερη προσθήκη θα βοηθούσε τον 
αλγόριθμο γιατί τώρα ο λόγος των ομάδων που έχουν αρκετά στοιχεία προς αυτών που έχουν ελάχιστα
είναι πολύ μικρός με αποτέλεσμα να δαπανούνται πόροι ώστε να σηκωθούν τα thread για να
γίνουν ελάχιστοι υπολογισμοί. 

\section*{Κώδικας}

\section*{Έλεγχος ορθότητας}

Για τον έλεγχο ορθότητας της συγκεκριμένης υλοποίησης γίνεται σύγκριση των αποτελεσμάτων 
του αλγορίθμου με τα αποτελέσματα της υλοποίησης ανοιχτού κώδικα που δόθηκε στην εκφώνηση. Για 
κάθε dataset που χρησιμοποιήθηκε πρώτα έγινε ο υπολογισμός του pagerank στην 
matlab και εξάχθηκαν τα αποτελέσματα σε ένα αρχείο, από το οποίο διαβάζεται το 
διάνυσμα του pagerank και συγκρίνεται στοιχείο με στοιχείο με την υλοποίηση 
που αναπτύχθηκε στα πλαίσια της εργασίας. 

Στον πίνακα 2να **βαλω λινκ** αναγράφεται το error για κάθε dataset. 



\end{document}

